\documentclass{article}
\usepackage[utf8]{inputenc}

\usepackage[framed,numbered,autolinebreaks,useliterate]{mcode}
\usepackage{url}
\setlength{\parindent}{0pt}
\setlength{\parskip}{18pt}

\usepackage{graphicx}

\title{\textbf{Numerical Analysis – Winter 2021}}
\author{\textbf{3180100675}}
\date{2021.12.2}

\begin{document}

\maketitle

\section{Problem 1:}  

Because $f(x)=-x^{3}-cos(x)$, $f^{'}(x)=-3x^{2}+sin(x)$, then refer to $P_{n}=P_{n-1}-\frac{f(P_{n-1})}{f^{'}(P_{n-1})}$, $p_{2}=-0.865684$. \\
And due to $f^{'}(0)=0$, it is strictly forbidden to using $p_{0}=0$ to iterate!

\section{Problem 2:}
(\romannumeral1)
Because $f(x)=b-\frac{1}{x}$, $f^{'}(x)=\frac{1}{x^{2}}$;
Refer to this problem description, $\epsilon_{k}=\frac{\frac{1}{b}-x_{k}}{\frac{1}{b}}$, and then $x_{k+1}=x_{k}-\frac{b-\frac{1}{x_{k}}}{\frac{1}{x^2}}=-bx_{k}^{2}+2x_{k}$, so \\
$|\epsilon_{k+1}|=|\frac{\frac{1}{b}-x_{k+1}}{\frac{1}{b}}|=$
$|\frac{x_k^2-\frac{2x_k}{b}+\frac{1}{b^{2}}}{\frac{1}{b^2}}|$
$=(\frac{x_k-\frac{1}{b}}{\frac{1}{b}})^2$
$=\epsilon_k^2$

(\romannumeral2)
We know $g(x)=2x-bx^2$, so $g(x)$ is symmetry on line $x=\frac{1}{b}$. There i just need to proof it will converge on $x\in(0,\frac{1}{b})$. \\

It is obvious that $g(x)-x=x-bx^2>0, x\in(0,\frac{1}{b})$, so $x<g(x)<\frac{1}{b}$. At the same time: $x_k=g(x_{k-1})$, therefore ${x_k}$ is a monotonic increasing sequence and infinitely approximate the upper bound $\frac{1}{b}$, and refer to the definition of array list, finally $x_k$ will converge to the real root $\frac{1}{b}$. \\

Below is a figure that illustrates this idea, too!
\begin{figure}[htbp]
\centering
\includegraphics[width=5cm]{HW2-P2.jpg}
\caption{Illumination of convergence}
\end{figure}

\section{Problem 3:}

Appendix A is the Newton's Method Algorithm, i have debugged this code and get the right solution for these two nonlinear equations.\\
Then applying this code and using $\textbf{x}^{(0)}=0$, finally i get: \\
a. $\textbf{x}^{(2)}=[4.3509, 18.4912, -19.8421]^T$; \\
b. $\textbf{x}^{(2)}=[0.5002, 0.2508, -0.5174]^T$;

\section{Problem 4:} 
Appendix B is the Steepest Descent Algorithm. I make good use of "solve" function in MATLAB, and acquire their exact solutions, then i choose corresponding initial values to ensure solving every solutions! Their roots are: \\
a. (*) There exists 4 plural solutions which i didn't list below. \\
root1 = [1.0380, 1.0803, 0.9303]; \\
root2 = [-8.4400, -7.9400, -19.1438]; \\
b. \\
root1 = [0.2102, 0.3598, -1.4070]; \\
root2 = [0.8407, -0.3583, 1.4079]; \\
root3 = [0.8998, -0.9967, 0.5079]; \\
root4 = [0.4962, 0.9967, -0.5076]; \\
	 

\newpage
\appendix
\section{Appendix A}

\begin{lstlisting}
clear; clc; close all;
%% Functions define
global f jacob;
syms x1 x2 x3;
% a.
f1 = x1 ^ 2 + x2 - 37; % x(2) = 4.3509   18.4912  -19.8421
f2 = x1 - x2 ^ 2 -5;
f3 = x1 + x2 + x3 - 3;
% b.
% f1 = 3 * x1 - cos(x2 * x3) - 0.5; % x(2) = 0.5002    0.2508   -0.5174
% f2 = 4 * x1 ^ 2 - 625 * x2 ^ 2 + 2 * x2  - 1;
% f3 = exp(-x1 * x2) + 20 * x3 + (10 * pi - 3) / 3;
jacobian = [diff(f1,x1), diff(f1,x2), diff(f1,x3);
    diff(f2,x1), diff(f2,x2), diff(f2,x3);
    diff(f3,x1), diff(f3,x2), diff(f3,x3)];
f = matlabFunction([f1;f2;f3]);
jacob = matlabFunction(jacobian);

%% Main
p0 = [0;0;0];
MaxIteration = 100;
Tolerance = 10^-10;
oldP = p0;
for i = 1:MaxIteration
    fprintf("[No.%d Iteration]: F(", i); disp(oldP'); fprintf(") = ");
    disp(F(oldP)');
    if max(abs(F(oldP))) < Tolerance % using infinity norm to stop!
        fprintf("It costs me %d iterations to find the root:\n", i);
        disp(oldP');
        break;
    end
    newP = NewTon(oldP);
    oldP = newP;
end

%% My functions
function Y = F(X)
    global f;
    Y = f(X(1),X(2),X(3));
end
function M = Jacob(X)
    global jacob;
    M = jacob(X(1),X(2)); % (i)
%     M = jacob(X(1),X(2),X(3)); % (ii)
end
function Xk = NewTon(Xk_1)
    Xk = Xk_1 - pinv(Jacob(Xk_1)) * F(Xk_1);
end
\end{lstlisting}

\section{Appendix B}
\begin{lstlisting}
clear; clc; close all;
%% Functions define
global f dg;
syms x1 x2 x3; 
% a.
f1 = 15 * x1 + x2 ^ 2 - 4 * x3 - 13; 
f2 = x1 ^ 2 + 10 * x2 - x3 - 11;
f3 = x2 ^ 3 - 25 * x3 + 22;
% b.
% f1 = 10 * x1 - 2 * x2 ^ 2 + x2 - 2 * x3 - 5;
% f2 = 8 * x2 ^ 2 + 4 * x3 ^ 2  - 9;
% f3 = 8 * x2 * x3 + 4;
g = f1 ^ 2 + f2 ^ 2 + f3 ^ 2;
DELTA_g = [diff(g,x1); diff(g,x2); diff(g,x3)];
f = matlabFunction([f1;f2;f3]);
dg = matlabFunction(DELTA_g);

%% Main
p0 = [1;1;1]; % (i) 1.0380    1.0803    0.9303
% p0 = [-8.44;-7.94;-19.1438]; % (i) -8.4414    -7.9401    -19.1438
% p0 = [1;0;-1]; % (ii) 0.2102    0.3598   -1.4070
% p0 = [1;0;1]; % (ii) 0.8407   -0.3583    1.4079
% p0 = [0;-1;1]; % (ii) 0.8998   -0.9967    0.5079
% p0 = [0;1;-1]; % (ii) 0.4962    0.9967   -0.5076
MaxIteration = 100;
Tolerance = 0.05;
oldP = p0;
for i = 1:MaxIteration
    fprintf("[No.%d Iteration]: F(", i); disp(oldP'); fprintf(") = ");
    disp(F(oldP)');
    if max(abs(F(oldP))) < Tolerance % using infinity norm to stop!
        fprintf("It costs me %d iterations to find the root:\n", i);
        disp(oldP');
        break;
    end
    newP = Descent(oldP);
    oldP = newP;
end

%% My functions
function Y = F(X)
    global f;
    Y = f(X(1),X(2),X(3));
end
function M = DG(X)
    global dg;
    M = dg(X(1),X(2),X(3));
end
function Xk = Descent(Xk_1)
    alpha = 0.001;
    Xk = Xk_1 - alpha * DG(Xk_1);
end
\end{lstlisting}

\end{document}

