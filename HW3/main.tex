\documentclass{article}
\usepackage[utf8]{inputenc}

\usepackage[framed,numbered,autolinebreaks,useliterate]{mcode}
\usepackage{url}
\setlength{\parindent}{0pt}
\setlength{\parskip}{18pt}

\usepackage{graphicx}

\title{\textbf{Numerical Analysis – Winter 2021}}
\author{\textbf{3180100675}}
\date{2021.12.13}

\begin{document}

\maketitle

\section{Problem 1:}  
According to infinity norm definition: \\
\begin{equation}
    ||A||_{\infty} = \max_{1\leq i\leq n} \sum_{j=1}^n |a_{ij}|
\end{equation}
\\
It is obviously that $||x-\widetilde{x}||_{\infty} = max(|x-\widetilde{x}|)$, so: \\
\\
a. \\
$||x-\widetilde{x}||_{\infty} = 0.5$ \\
$||A\widetilde{x}-b||_{\infty} = 0.3$ \\
\\
b. \\
$||x-\widetilde{x}||_{\infty} = 0.9$ \\
$||A\widetilde{x}-b||_{\infty} = 0.27$ \\

\newpage
\section{Problem 2:}
Refer to "Lec05.pdf" Page 23 theorem 1: \\
\begin{equation}
    ||A||_{2} = [\rho(A^{T}A)]^{\frac{1}{2}}
\end{equation}
\\
And because $A$ is a symmetric matrix, $A^{T} = A$, i only need to prove:
\begin{equation}
    \rho(A) = [\rho(A^{2})]^{\frac{1}{2}}
\end{equation}
\\
If $\lambda$ is a eigenvalue of $A$, $Ax=\lambda x$, then $A^2x=AAx=A\lambda x=\lambda^2x$, so $\lambda^2$ is a corresponding eigenvalue of $A^2$. Therefore, $\rho(A^2)=max|\lambda^2|=max^2|\lambda|=\rho^2(A)$, equation(3) is proved. So $||A||_2=\rho(A)$.

\section{Problem 3:}
Appendix A is the Gaussian elimination Algorithm, i have debugged this code and get the approximating solution for these two linear equations. Also i test a non-unique equation and raise ERROR properly.\\
Then applying this program and using scaled partial pivoting, finally i get: \\
\\
a. $\widetilde{x}=[10, 1]^T$; \\
\\
b. $\widetilde{x}=[-0.0559, 9.9860, 0.1429]^T$;

\newpage
\section{Problem 4:} 
Appendix B is the Jacobi iterative Algorithm. I firstly using "triu", "tril" and "diag" internal functions in Matlab to get U, L and D; then Jacobi iteration is been applied so that x is gradually approximating the real solution, finally i also calculate the equation to ensure my algorithm is right. \\ 
After debugging, my first three iteration results are : \\
\\
a. \\
$x^{(1)} = [1.2500, -1.3333, 0.2000]$; \\
$x^{(2)} = [1.6333, -0.9833, 0.2333]$; \\
$x^{(3)} = [1.5542, -0.8667, -0.0600]$; \\
\\
b. \\
$x^{(1)} = [-2, 2, 0]$; \\
$x^{(2)} = [-1, 1, -1]$; \\
$x^{(3)} = [-1.7500, 1.7500, -0.5000]$; \\

\section{Problem 5:}
Appendix C is the Gaussian-Seidel Algorithm. This time i can not utilize $x^{(k)} = D^{-1}(L+U)x^{(k-1)}+D^{-1}b$ to update in single step, what i did is to loop over all the roots and update every root using last updated root! \\
After debugging, my solutions are: \\
\\
a. \\
1. Gaussian-Seidel Algorithm: $\widetilde{root}=[0.0352, -0.2368, 0.6579]$ (7 iterations) \\
2. Jacobi iterative Algorithm: $\widetilde{root}=[0.0351, -0.2369, 0.6578]$ (10 iterations) \\
\\
b. \\
1. Gaussian-Seidel Algorithm: $\widetilde{root}=[0.9957, 0.9578, 0.7915]$ (6 iterations) \\
2. Jacobi iterative Algorithm: $\widetilde{root}=[0.9957, 0.9579, 0.7916]$ (4 iterations) \\
\\
These two algorithm both get a good approximation of real root, and Gaussian-Seidel is converge faster than Jacobi iterative algorithm.

\newpage
\appendix
\section{Appendix A}

\begin{lstlisting}
clc; clear;
main();
%% My functions
function main()
    % a Problem
    A = [0.03 58.9;
        5.31 -6.10];
    b = [59.2; 47.0];
    % b Problem
%     A = [3.03 -12.1 14;
%         -3.03 12.1 -7;
%         6.11 -14.2 22];
%     b = [-119; 120; -139];
    result = Gaussion_elimination(A, b);
    fprintf("Result :");
    disp(result);
end
function root = Gaussion_elimination(Mat, b)
    Aug_M = [Mat, b];
    ROW = size(Mat,1); % normally COL == ROW
    COL = size(Mat,2);
    for i = 1:ROW-1
        % Scaled Matrix [Normalization]
        Aug_M = Aug_M ./ max(abs(Aug_M(:,1:COL)), [], 2);
        [pivoted_V, index] = max(Aug_M(i:end, i)); % Pivoted Value
        if (pivoted_V == 0)
            fprintf("[ERROR]: No unique solution exists!\n");
            return;
        end
        index = index + i - 1;
        if (index ~= i) % Swap ith row and the indexth row
            temp = Aug_M(i, :);
            Aug_M(i, :) = Aug_M(index, :);
            Aug_M(index, :) = temp;
        end
        % Substraction
        subM = zeros(ROW, COL+1);
        subM(i+1:end,:) = Aug_M(i, :) ./pivoted_V .* Aug_M(i+1:end,i);
        Aug_M = Aug_M - subM;
    end
    % Backward subsitition
    root = zeros(1, COL);
    for i = 1:ROW
        j = ROW - i + 1;
        if (Aug_M(j,j) == 0)
            fprintf("[ERROR]: No unique solution exists!\n");
            return;
        end
        root(j) = (Aug_M(j,end) - sum(root(j+1:end) .* Aug_M(j,j+1:COL))) / Aug_M(j,j);
    end
end
\end{lstlisting}

\section{Appendix B}
\begin{lstlisting}
clc; clear; close all;
%% Main
main();
%% My functions
function main()
    % a Problem
%     A = [4 1 -1;
%         -1 3 1;
%         2 2 5];
%     b = [5; -4; 1];
    % b Problem
    A = [-2 1 1/2;
        1 -2 -1/2;
        0 1 2];
    b = [4; -4; 0];

    result = Jacobi_iterative(A,b);
    fprintf("Result :");
    disp(result');
end

function root = Jacobi_iterative(A,b) % A = D - L - U
    D = diag(diag(A));
    L = -tril(A,-1);
    U = -triu(A,1);
    
    n = size(A,1);
    MaxIteration = 100;
    TOL = 0.001;
    oldRoot = zeros(n,1); % Init
    newRoot = zeros(n,1);
    for i = 1:MaxIteration
        newRoot = pinv(D) * (L + U) * oldRoot + pinv(D) * b;
        fprintf("No.%d iteration: x(%d) =", i, i);
        disp(newRoot');
        if (inf_norm(oldRoot - newRoot) / inf_norm(newRoot) < TOL)
            fprintf("After %d iterations, i found the root!\n", i);
            break;
        end
        oldRoot = newRoot; % Update
    end
    root = newRoot;
    
    if (i == MaxIteration)
        fprintf("[Warn]: Maximum number of iterations exceeded\n");
    end
    fprintf("A * x - b =");
    disp((A * root - b)');
end

function ret = inf_norm(V)
    ret = max(abs(V));
end
\end{lstlisting}

\section{Appendix C}
\begin{lstlisting}
clc; clear; close all;
%% Main
main();
%% My functions
function main()
    % a Problem
%     A = [3 -1 1;
%         3 6 2;
%         3 3 7];
%     b = [1; 0; 4];
    % b Problem
    A = [10 -1 0;
        -1 10 -2;
        0 -2 10];
    b = [9; 7; 6];
    result = Jacobi_iterative(A,b);
    fprintf("Jacobi Iterative Result :");
    disp(result');
    result = Gauss_Seidel(A,b);
    fprintf("Gauss-Seidel Result :");
    disp(result');
end
function root = Gauss_Seidel(A,b) % A = D - L - U
    D = diag(diag(A));
    L = -tril(A,-1);
    U = -triu(A,1);
    Update_A = pinv(D) * (L + U); % (n,n)
    Update_b = pinv(D) * b; % (n,1)
    n = size(A,1);
    MaxIteration = 100;
    TOL = 0.001;
    oldRoot = zeros(n,1); % Init
    newRoot = zeros(n,1);
    for i = 1:MaxIteration
        for j = 1:n
            newRoot(j) = sum(newRoot' .* Update_A(j,:)) + Update_b(j);
        end
        fprintf("No.%d iteration: x(%d) =", i, i);
        disp(newRoot');
        if (inf_norm(oldRoot - newRoot) / inf_norm(newRoot) < TOL)
            fprintf("After %d iterations, i found the root!\n", i);
            break;
        end
        oldRoot = newRoot; % Update
    end
    root = newRoot;
    if (i == MaxIteration)
        fprintf("[Warn]: Maximum number of iterations exceeded\n");
    end
    fprintf("A * x - b =");
    disp((A * root - b)');
end
function root = Jacobi_iterative(A,b) % A = D - L - U
    D = diag(diag(A));
    L = -tril(A,-1);
    U = -triu(A,1);
    
    n = size(A,1);
    MaxIteration = 100;
    TOL = 0.001;
    oldRoot = zeros(n,1); % Init
    newRoot = zeros(n,1);
    for i = 1:MaxIteration
        newRoot = pinv(D) * (L + U) * oldRoot + pinv(D) * b;
        fprintf("No.%d iteration: x(%d) =", i, i);
        disp(newRoot');
        if (inf_norm(oldRoot - newRoot) / inf_norm(newRoot) < TOL)
            fprintf("After %d iterations, i found the root!\n", i);
            break;
        end
        oldRoot = newRoot; % Update
    end
    root = newRoot;
    
    if (i == MaxIteration)
        fprintf("[Warn]: Maximum number of iterations exceeded\n");
    end
    fprintf("A * x - b =");
    disp((A * root - b)');
end

function ret = inf_norm(V)
    ret = max(abs(V));
end
\end{lstlisting}


\end{document}

